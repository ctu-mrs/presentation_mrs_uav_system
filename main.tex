%----------------------------------------------------------------------------------------
%	PACKAGES AND THEMES
%----------------------------------------------------------------------------------------

\documentclass[aspectratio=169]{beamer}

\mode<presentation> {

  \usetheme{Boadilla} % light

  \usecolortheme{seahorse} % light

  \input{./cvut_colors.tex}

  % \setbeamertemplate{footline} % remove the footer line
  % \setbeamertemplate{footline}[page number] % replace the footer line with simple numbers

  \setbeamertemplate{navigation symbols}{}
  \setbeamertemplate{bibliography item}{\insertbiblabel} % removing the navigation symbols

}

%%{ Docu HEAD

\usepackage{graphicx} % Allows including images
\usepackage{booktabs} % Allows the use of \toprule, \midrule and \bottomrule in tables
\usepackage{multimedia}
\newcommand{\superfill}{\vskip0pt plus 1filll}

\usepackage{isotope}
\usepackage{animate}

\usepackage[export]{adjustbox}

\usepackage{graphicx}
\usepackage{setspace}
\usepackage{epstopdf}
\usepackage{float}
\usepackage{multirow,tabularx,makecell}

\usepackage{pdfpcnotes}

\usefonttheme{professionalfonts}
\usepackage{amsmath,amsfonts,amssymb,bm}

\usepackage[backend=bibtex,defernumbers=true,style=ieee,sorting=none,sortcites=false]{biblatex}

\renewcommand*{\bibfont}{\normalfont\tiny}

% Print labelnumbers with suffixes, adjust secondary labelnumber 2/2
\DeclareFieldFormat{labelnumber}{%
  \ifkeyword{mine}
  {\ifkeyword{core}
  {{\number\numexpr#1}}%
  {{\number\numexpr#1}}%
  }%
  {#1}%
  }

  \DeclareCiteCommand{\tabcite}%[\mkbibbrackets]
  {\usebibmacro{cite:init}%
  \usebibmacro{prenote}}
  {\usebibmacro{citeindex}%
  \usebibmacro{cite:comp}}
  {}
  {\usebibmacro{cite:dump}%
  \usebibmacro{postnote}}

  % {{\number\numexpr#1-\value{bbx:primcount}}a}

  %%{ fullcite box

  \usepackage{tcolorbox}

  \definecolor{light-gray}{gray}{0.95}
  \newcommand{\fullciteinbox}[2]{%

    \DeclareCiteCommand{\fullcite}
    {\usebibmacro{prenote}}
    {\clearfield{addendum}%
    \clearfield{issn}%
    \usedriver
    {\defcounter{minnames}{6}%
    \defcounter{maxnames}{6}}
    {\thefield{entrytype}}}
    {\multicitedelim}
    {\usebibmacro{postnote}}

    %\vspace{3em}%
    %\raisebox{3em}[3em][3em]{%
    % \vspace{-0.2em}
    % \begin{tcolorbox}[width=\textwidth,colback={light-gray},title={}]%
    \begin{block}{}
      \begin{minipage}[t]{0.07\linewidth}%
        \raggedright%
        \scriptsize \cite{#1}%
      \end{minipage}%
      \begin{minipage}[t]{0.93\linewidth}%
        \scriptsize \fullcite{#1}%
        \ifx&#2&
        \else
        \\
        \url{#2}
        \fi
      \end{minipage}%
      % \end{tcolorbox}%
    \end{block}
    %}%
    \vspace{-0.3em}
    }%

    %%}

    \addbibresource{main.bib}

    \defbibenvironment{favoritebib}
  {\itemize}
    {\enditemize}
  {\item}
    \usepackage{siunitx}
    \DeclareSIUnit \parsec {pc}
    \DeclareSIUnit \electronvolt {eV}
    \DeclareSIUnit \pixel {px}
    \DeclareSIUnit \arcmin {arcmin}
    \DeclareSIUnit \erg {erg}
    \DeclareSIUnit \joul {J}
    \DeclareSIUnit \Bq {Bq}

    \usepackage{enumitem} % To enable enumerate with letters (a, b, c...)
    \newlist{alphalist}{enumerate}{1}
    \setlist[alphalist]{label=\alph*)}
    \setlist[itemize]{label=\textbullet}

    \usepackage{cellspace}
    \newcolumntype{D}{>{\hfill}N{3}{2}<{\hfill}}
    \newcommand*\cellspacelimit[2]{\setlength{\cellspacetoplimit}{#1}\setlength{\cellspacebottomlimit}{#2}}

    % figures
    \usepackage{wrapfig}
    % \usepackage[font={footnotesize}]{caption}
    \usepackage[font={small}]{caption}
    % \usepackage{subcaption}

    % subfloat
    \usepackage{subfig}
    % \usepackage[export]{adjustbox}

    \usepackage{color}
    \usepackage{url}

    %%{ tikz

    \usepackage{tikz}
    \usepackage{pgfplots}
    \pgfplotsset{compat=1.14}
    \usetikzlibrary{backgrounds,arrows,automata,shapes,positioning,calc,through,spy,shapes,shapes.geometric,shapes.multipart,fit,patterns,fadings}
    \pgfdeclarelayer{background}
    \pgfdeclarelayer{foreground}
    \pgfsetlayers{background,main,foreground}

    \input{./fig/tikz/tikz.tex}

    \tikzset{
      imgletter/.style={
        rectangle,
        inner sep=2pt,
        rounded corners=.1em,
        text=black,
        minimum height=1em,
        text centered,
        fill=white,
        fill opacity=1.0,
        text opacity=1,
        anchor=south west,
      },
    }

    \input{./tikz-templates_light.tex}

    %%}

    \usepackage{pdfpc-movie}
    \newcommand{\mymovie}[3][]{\pdfpcmovie[#1]{#2}{#3}}
    % \newcommand{\mymovie}[3][]{\movie[#1]{#2}{#3}}

            %%{ CHECKMARK IN TIKZ

            \def\checkmark{\tikz\fill[scale=0.4](0,.35) -- (.25,0) -- (1,.7) -- (.25,.15) -- cycle;}

            %%}

    \newcommand{\strong}[1]{\textbf{#1}}
    \newcommand{\coord}[1]{\textbf{#1}}
    \newcommand{\norm}[1]{\left\lvert#1\right\rvert}
    \newcommand{\m}[1]{\ensuremath{\mathbf{#1}}}
    \newcommand\numberthis{\addtocounter{equation}{1}\tag{\theequation}}
    \newcommand{\corrected}[1]{{\color{black} {#1}}}
    % \newcommand{\comment}[1]{{\color{blue} {#1}}}
    \newcommand{\add}[1]{{\color{green} {#1}}}
    \newcommand{\todo}[1]{{\color{red} TODO {#1}}}
    \newcommand{\updated}[1]{{\color{blue} {#1}}}
    \newcommand{\reffig}[1]{Fig.~\ref{#1}}
    \newcommand{\refalg}[1]{Alg.~\ref{#1}}
    \newcommand{\refsec}[1]{Sec.~\ref{#1}}
    \newcommand{\reftab}[1]{Table~\ref{#1}}
    \newcommand{\real}{\mathbb{R}}
    \newcommand{\red}[1]{{\color{red} #1}}
    \newcommand{\minus}{\scalebox{0.75}[1.0]{$-$}}
    \newcommand{\plus}{\scalebox{0.8}[0.8]{$+$}}
    \newcommand{\figvspace}{\vspace{-1em}} % this may eventually do something, so far just a placeholder

    % \usepackage{pgf}
    % \logo{\pgfputat{\pgfxy(0,5)}{\pgfbox[right,base]{\includegraphics[height=0.8cm]{}}}}
    % \newcommand{\nologo}{\setbeamertemplate{logo}{}}

    % \usepackage{eso-pic}
    % \newcommand\AtPagemyUpperLeft[1]{\AtPageLowerLeft{\put(\LenToUnit{0.66\paperwidth},\LenToUnit{0.904\paperheight}){#1}}}
    % \AddToShipoutPictureFG{
    %   \AtPagemyUpperLeft{{\includegraphics[height=0.85cm,keepaspectratio]{fig/logo_ctu_fee_mrs_blue.png}}}
    % }
    % \newcommand{\AddToShipoutPictureFG}{\setbeamertemplate{logo}{}}

    %%}

    %%{ TITLE PAGE


    \title[MRS UAV System]{MRS UAV System for Real-world Testing and Development with Multirotor Aerial Vehicles}
    \subtitle{From control theory to practical use}
    \author[Tomas Baca]{{Tomas Baca} \\
    }
    \institute[CTU in Prague] {
      \begin{small}
        Multi-Robot Systems group, Faculty of Electrical Engineering\\
        Czech Technical University in Prague
      \end{small}}

    \date[July 29th, 2024]{}

    \titlegraphic{\includegraphics[width=5cm]{fig/logo_ctu_fee_mrs_blue.png}}

    \begin{document}

    \begin{frame}

      \titlepage % Print the title page as the first slide

    \end{frame}

    % \nologo

    %%}

  \begin{frame}
  \frametitle{Join the SLACK! Please!}
  
    \begin{center}
      \includegraphics[width=0.8\textheight]{./fig/SLACK_qr.png} 
    \end{center}
  
  \end{frame}

\begin{frame}
  \frametitle{Outline}
  \tableofcontents
\end{frame}

\section{UAV Experimentation in MRS}

%%{ kumar

\begin{frame}
\frametitle{Motivation --- Awesome multi-UAV videos}

\begin{block}{2012, University of Pennsylvania, prof. Vijay Kumar}
  \begin{center}
    \mymovie[autostart,loop]{
      \includegraphics[width=0.7\textwidth]{./fig/kumar_thumbnail.jpg}
    }{./videos/kumar.mp4}\\
    Video: \url{https://youtu.be/4ErEBkj_3PY}
  \end{center}
\end{block}

\end{frame}

%%}

%%{ UAV Experimentation in MRS

\begin{frame}
  \frametitle{UAV Experimentation in MRS (IMR) --- 2012--2015}

  \vspace{-0.5em}

  \only<+>{\begin{block}{$\approx$ 2012, designing custom PCBs for controllers}
    \centering
    \includegraphics[width=0.48\textwidth]{fig/photos/pcb1.jpg}
    \includegraphics[width=0.48\textwidth]{fig/photos/first_UAV.jpg}
  \end{block}}

  \only<+>{\begin{block}{$\approx$ 2014, Model Predictive Control on embedded hardware}
    \centering
    \includegraphics[width=0.48\textwidth]{fig/photos/pcb2.jpg}
    \includegraphics[width=0.48\textwidth]{fig/photos/prase.jpg}

    \fullciteinbox{baca2016embedded}{}
  \end{block}}

  \only<+>{\begin{block}{Embedded Model Predictive Control}
    \begin{center}

      \mymovie[autostart,loop]{
        \includegraphics[width=0.7\textwidth]{fig/photos/embedded_mpc_thumbnail.jpg}
      }{videos/embedded_mpc.mp4}\\
      Video: \url{http://youtu.be/AXI_rkQRBaE}

    \end{center}
  \end{block}}

  \only<+>{\begin{block}{Embedded Model Predictive Control}
    \begin{center}
      \mymovie[autostart,loop,start=15]{
        \includegraphics[width=0.7\textwidth]{fig/photos/mmar_vojta_thumbnail.jpg}
      }{videos/mmar_vojta.mp4}\\
      Video: \url{http://youtu.be/9Bpm4J31CgE}
    \end{center}
  \end{block}}

\end{frame}

\begin{frame}
  \frametitle{UAV Experimentation in MRS --- 2012--2015}

  \begin{itemize}
    \item Custom and purpose-built hardware
    \item Embedded software
    {\color{red} \item not scalable! bottlenecks everywhere
    \item embedded programming was \emph{cumbersome}
    \item simulating was difficult (custom-built hardware did not help)
    \item system maintenance was not flexible
    \item low modularity
    \item limited availability of sensors and peripheries}
  \end{itemize}

  \begin{block}{Publications}
    \cite{saska2013adhoc}, \cite{chudoba2014localization}, \cite{baca2016embedded}, \cite{saska2017documentation}, \cite{chudoba2016exploration}, \cite{saska2016formations}, \cite{spurny2016complex}, \cite{saska2017system}
  \end{block}

\end{frame}

%%}

%%{ Sim-to-real

\begin{frame}
\frametitle{Simulation-to-reality gap}

  \only<1>{
    \includegraphics[width=1.0\textwidth]{./fig/sim_to_real/sim_to_real_1.pdf}
  }

  \only<2>{
    \includegraphics[width=1.0\textwidth]{./fig/sim_to_real/sim_to_real_2.pdf}
  }

  \only<3>{
    \includegraphics[width=1.0\textwidth]{./fig/sim_to_real/sim_to_real_3.pdf}
  }

  \only<4>{
    \includegraphics[width=1.0\textwidth]{./fig/sim_to_real/sim_to_real_4.pdf}
  }

  \only<5>{
    \includegraphics[width=1.0\textwidth]{./fig/sim_to_real/sim_to_real_5.pdf}
  }

  \only<6>{
    \includegraphics[width=1.0\textwidth]{./fig/sim_to_real/sim_to_real_6.pdf}
  }

\end{frame}

%%}

%%{ CURRENT SYSTEM diagram

\begin{frame}
\frametitle{Current System Diagram}

\begin{center}
  \includegraphics[width=0.8\textwidth]{./fig/new_diagram.png}
\end{center}

\end{frame}

%%{ Nowadays
\begin{frame}
  \frametitle{UAV Experimentation in MRS --- 2016--now}

  \begin{block}{A typical pipeline structure revolves around a Linux computer.}
    \begin{figure}
      \includegraphics[width=0.6\textwidth]{fig/x500_labeled.pdf}
    \end{figure}
  \end{block}

\end{frame}
%%}

%%}

            %%{ ROS -- Robot Operating System

            \begin{frame}
              \frametitle{ROS -- Robot Operating System}

              \begin{itemize}
                \item middleware allowing communication between programs
                \item integrates with \texttt{C++}, Python, Bash and Zshell
                \item makes the transition from \emph{Matlab} to reality bearable
                \item supported by sensor manufacturers
                \item integration through the Linux terminal
                \item out of the box: time and clock management, logging, recording onboard data, visualization and plotting, parameter loading, static and dynamic transformations, etc.
                \item integrates to robotic simulators: Gazebo, Coppelia (V-REP), AirSim
              \end{itemize}

              \begin{figure}
                \includegraphics[width=0.3\textwidth]{fig/ros_logo.jpg}
              \end{figure}

            \end{frame}

            %%}

%%{ Implementation diagram

\begin{frame}
\frametitle{Modularity and abstractions}

\begin{columns}[c]

\column{0.48\textwidth} % Left column and width

\begin{itemize}
  \item Promotes collaborative development.
  \item Allows component substitution.
  \item Incentivises to build abstract interfaces.
  \item Allows for distributed execution.
  \item Makes introspection easier.
\end{itemize}

\column{0.48\textwidth} % Right column and width

\begin{block}{Modular system architecture}
  \begin{adjustbox}{max totalsize={1.0\textwidth}{.65\textheight}, center}
    \input{./fig/tikz/implementation_diagram.tex}
  \end{adjustbox}
\end{block}

\end{columns}

\end{frame}

\begin{frame}
\frametitle{Full implementation diagram}

  \includegraphics[width=1.0\textwidth]{./fig/full_diagram.jpg}

\end{frame}

%%}

%% | ------------ Control pipeline - implementation ----------- |

\section{UAV Control pipeline}
\subsection{Control -- math point of view}

%%{ UAV control pipepline

\begin{frame}
  \frametitle{Control theory point of view}

  \begin{figure}
    \begin{adjustbox}{max totalsize={1.0\textwidth}{.85\textheight}, center}
      \input{fig/tikz/pipeline_diagram.tex}
    \end{adjustbox}
  \end{figure}

  \fullciteinbox{baca2021mrs}{}

\end{frame}

%%}

\subsection{Control -- implementation point of view}

%%{ From theory to practical experiments

%%{ Linear Kalman filter

\begin{frame}
  \frametitle{From theory to practical experiments}

  \begin{columns}[c]

    \onslide<1->{\column{0.48\textwidth} % Left column and width
    \begin{block}{Linear Kalman Filter $\approx$ 10 lines}
      \begin{adjustbox}{max totalsize={1.0\textwidth}{.85\textheight}, center}
        \input{fig/tikz/kalman.tex}
      \end{adjustbox}
    \end{block}}

    \onslide<2>{\column{0.48\textwidth} % Right column and width
    \begin{block}{Linear Kalman Filter in practice}
      \begin{itemize}
        \item tens of lines in Matlab
        \item $\ge$ 1000 lines of \texttt{C++} code
        \item Real-world fusion \& estimation\\
          $\ge$ 20 000 lines of \texttt{C++} code
      \end{itemize}
    \end{block}}

  \end{columns}

\end{frame}

%%}

%%{ Control

\begin{frame}
  \frametitle{From theory to practical experiments}

  \begin{columns}[c]

    \column{0.60\textwidth} % Left column and width
    \onslide<1->{\vspace{-1em}
\begin{block}{Geometric tracking controller $\approx$ 10 lines}
\vspace{-1em}
\tiny\begin{align}
\mathbf{f}_d &= \minus m_e\mathbf{k}_p\circ \mathbf{e}_p \minus m_e\mathbf{k}_v\circ \mathbf{e}_v \plus m_e\ddot{\mathbf{r}}_d \minus m_eg\mathbf{\hat{e}}_3 \minus \mathbf{d}_w\circ \left[\begin{smallmatrix}
1\\
1\\
0
\end{smallmatrix}\right] \minus \mathbf{d}_b\circ \left[\begin{smallmatrix}
1\\
1\\
0
\end{smallmatrix}\right],\nonumber\\
\mathbf{e}_R &= \frac{1}{2}\left(\mathbf{R}_d^\intercal\mathbf{R} - \mathbf{R}^\intercal\mathbf{R}_d\right),\nonumber\\
\bm{\omega}_d &= \minus \mathbf{k}_R\circ \mathbf{e}_R \plus \bm{\omega}_j \minus \bm{\omega}_c,\nonumber
\end{align}
\vspace{-1em}
\end{block}}

\onslide<2->{
\vspace{-0.5em}
\begin{block}{MPC controller $\approx$ 100 lines}
\vspace{-1em}
\scriptsize{
\begin{align*}
  & \min_{\mathbf{u}_{[1:n]}}\nonumber
  & \frac{1}{2}\sum_{i=1}^{n-1}\left(\mathbf{e}^\intercal_{[i]}\mathbf{Q}\mathbf{e}_{[i]}\right) + \mathbf{e}^\intercal_{[n]}\mathbf{S}\mathbf{e}_{[n]}
\end{align*}\begin{align*}
  \text{s.t.}~ \mathbf{x}_{m[i]} &= \mathbf{A}_m\mathbf{x}_{m[i-1]} \plus \mathbf{B}_m\mathbf{u}_{[i]}, &\forall i &\in \{1, \hdots, n\}\nonumber\\
  \mathbf{x}_{m[i]} &\leq \mathbf{x}_{\mathrm{max}}, &\forall i &\in \{1, \hdots, n\}\nonumber\\
  \mathbf{x}_{m[i]} &\geq \minus \mathbf{x}_{\mathrm{max}}, &\forall i &\in \{1, \hdots, n\}\nonumber \\
  \mathbf{u}_{[i]} \minus \mathbf{u}_{[i\minus 1]} &\leq \dot{\mathbf{u}}_{\mathrm{max}} \Delta t, &\forall i &\in \{2, \hdots, n\}\nonumber\\
  \mathbf{u}_{[i]} \minus \mathbf{u}_{[i\minus 1]} &\geq \minus \dot{\mathbf{u}}_{\mathrm{max}} \Delta t, &\forall i &\in \{2, \hdots, n\}\nonumber
\end{align*}
}
\vspace{-1em}
\end{block}}

\column{0.38\textwidth} % Right column and width

\onslide<3>{\begin{block}{UAV Control in practice}
\begin{itemize}
\item $\ge$ 10 000 lines of \texttt{C++} code
\end{itemize}
\end{block}}

\end{columns}

\end{frame}

%%}

%%{ From theory to practical experiments

\begin{frame}
  \frametitle{From theory to practical experiments}

  \begin{columns}[c]

    \column{0.70\textwidth} % Left column and width
    \begin{block}{What needs to be solved outside of Matlab's sandbox?}
      \begin{itemize}
        \item UAV crashes are expensive, so don't crash
        \item Software runtime errors cause UAV crashes
        \item control references might not be feasible
        \item sensors can get disconnected during the flight
        \item takeoff and landing: \textbf{the most tricky part of the flight}
        \item mass (thus the model) can change during the flight
        \item controllers can be poorly tuned... handle instabilities
        \item acceleration and speed depend on the available sensors
        \item people are fallible, don't let them crash the drones
        \item not all states of UAV are allowed, even though controllers can reach them (upside down)
      \end{itemize}
    \end{block}

    \column{0.28\textwidth} % Right column and width
    \begin{block}{The failsafe core}
      \texttt{if (goingToCrash())}\\
      \texttt{~~~~dont();}
    \end{block}

  \end{columns}

\end{frame}

%%}

\begin{frame}
\frametitle{Pushing the chances --- 14 UAVs in a swarm}

  \begin{center}
    \mymovie[autostart,loop]{
      \includegraphics[width=0.9\textwidth]{./fig/swarm_14uavs_field_thumbnail.jpg}
    }{./videos/swarm_14uavs_field.mp4}\\
  \end{center}

\end{frame}

\begin{frame}
\frametitle{Pushing the chances too much}

\begin{center}
  \mymovie[autostart,loop,start=5]{
    \includegraphics[width=0.75\textwidth]{./fig/crashes_thumbnail.jpg}
  }{./videos/crashes.mp4}\\
  Video: \url{https://youtu.be/dT7b1j5Ij1I?t=148}
\end{center}

\end{frame}

%%}

%% | --------------------- MRS UAV System --------------------- |

%%{ System diagram

\begin{frame}

\frametitle{The MRS UAV System}

  \begin{block}{MRS UAV System block diagram}
    \begin{figure}
      \begin{adjustbox}{max totalsize={1.0\textwidth}{.65\textheight}, center}
        \only<1>{\input{./fig/tikz/pipeline_diagram.tex}}
        \only<2>{\input{./fig/tikz/pipeline_diagram_controller.tex}}
      \end{adjustbox}
    \end{figure}
  \end{block}

  \fullciteinbox{baca2021mrs}{}

\end{frame}

%%}

\subsection{Reference controllers}

%%{ Reference controllers

%%{ Heading-compliant control design

\begin{frame}
\frametitle{Reference controllers --- Attitude rate}

\begin{block}{\small Heading-compliant control design}
  \small Heading $\eta = \mathrm{atan2}\left(\mathbf{\hat{b}}_1^\intercal\mathbf{\hat{e}}_2, \mathbf{\hat{b}}_1^\intercal\mathbf{\hat{e}}_1\right) = \mathrm{atan2}\left(\mathbf{h}_{(2)}, \mathbf{h}_{(1)}\right)$ as the $4^{\mathrm{th}}$ DOF (instead of \emph{yaw)}.
\end{block}

\begin{columns}[c]

\column{0.48\textwidth} % Left column and width

\begin{block}{\small \emph{SO(3)} attitude controller \cite{lee2010geometric}}
  \begin{itemize}
    \item \scriptsize Custom heading-compliant desired orientation:
  \end{itemize}

  \begin{equation}
    \scriptsize
    \mathbf{R}_d = \left[\mathbf{\hat{b}}_{1d}, \mathbf{\hat{b}}_{2d}, \mathbf{\hat{b}}_{3d}\right],
  \end{equation}

  \begin{equation}
    \scriptsize
    \mathbf{b}_{1d} = \mathbf{O}(\mathbf{P}^\intercal\mathbf{O})^{-1}\mathbf{P}^\intercal\,\mathbf{\hat{h}}_d,\ \mathbf{\hat{b}}_{1d} = \frac{\mathbf{b}_{1d}}{\|\mathbf{b}_{1d}\|},
  \end{equation}

  \begin{equation}
    \scriptsize
    \hat{\mathbf{b}}_{2d} = \mathbf{\hat{b}}_{3d}\times \mathbf{\hat{b}}_{1d}.
  \end{equation}

\end{block}

\column{0.48\textwidth} % Right column and width

  \vspace{-1em}

  \begin{figure}
    \centering
    \includegraphics[width=0.90\textwidth]{fig/sketch/coordinate_frames_platform.pdf}
    % \caption{
    %   The image depicts the world frame $\mathcal{W}$ = $\{\mathbf{\hat{e}}_1$, $\mathbf{\hat{e}}_2$, $\mathbf{\hat{e}}_3\}$ in which the 3D position and the orientation of the \ac{UAV} body is expressed.
    %   The body frame $\mathcal{B}$ = $\{\mathbf{\hat{b}}_1$, $\mathbf{\hat{b}}_2$, $\mathbf{\hat{b}}_3\}$ relates to $\mathcal{W}$ by the translation $\mathbf{r} = \left[x, y, z\right]^{\intercal}$ and by rotation $\mathbf{R}^{\intercal}$.
    %   The \ac{UAV} heading vector $\mathbf{h}$, which is a projection of $\hat{\mathbf{b}}_1$ to the plane $span\left(\mathbf{\hat{e}}_1, \mathbf{\hat{e}}_2\right)$, forms the heading angle $\eta = \mathrm{atan2}\left(\mathbf{\hat{b}}_1^\intercal\mathbf{\hat{e}}_2, \mathbf{\hat{b}}_1^\intercal\mathbf{\hat{e}}_1\right) = \mathrm{atan2}\left(\mathbf{h}_{(2)}, \mathbf{h}_{(1)}\right)$.
    %   }
  \end{figure}

\end{columns}

\fullciteinbox{baca2021mrs}{}

\end{frame}

%%}

\begin{frame}
\frametitle{Reference controllers --- Desired force and heading rate}

\begin{columns}[c]

\column{0.48\textwidth} % Left column and width

\begin{block}{\small Geometric tracking on \emph{SE(3)} \cite{lee2010geometric}}
  \begin{equation}
    \scriptsize
    \begin{split}
      \mathbf{f}_d = &\overbrace{-m_e\mathbf{k}_p\circ \mathbf{e}_p}^{\begin{tabular}{c}
        \tiny position\\
        \tiny feedback
      \end{tabular}} + \overbrace{-m_e\mathbf{k}_v\circ \mathbf{e}_v}^{\begin{tabular}{c}
        \tiny velocity\\
        \tiny feedback
      \end{tabular}} + \overbrace{m_e\ddot{\mathbf{r}}_d}^{\begin{tabular}{c}
        \tiny reference\\
        \tiny feedforward
      \end{tabular}} + \\
      & \underbrace{m_eg\mathbf{\hat{e}}_3}_{\begin{tabular}{c}
        \tiny gravity\\
        \tiny compensation
      \end{tabular}} + \underbrace{-\mathbf{d}_w\circ \left[\begin{smallmatrix}
        1\\
        1\\
        0
      \end{smallmatrix}\right]}_{\begin{tabular}{c}
        \tiny world disturbance\\
        \tiny compensation
      \end{tabular}} + \underbrace{-\mathbf{d}_b\circ \left[\begin{smallmatrix}
        1\\
        1\\
        0
      \end{smallmatrix}\right]}_{\begin{tabular}{c}
        \tiny body disturbance\\
        \tiny compensation
      \end{tabular}},
    \end{split}
  \end{equation}
\end{block}

\column{0.48\textwidth} % Right column and width

\begin{block}{\small Linear MPC}
  \begin{equation}
    \scriptsize
    \begin{split}
      \mathbf{f}_d = &\overbrace{m_e\ddot{\mathbf{r}}_d}^{\begin{tabular}{c}
        \tiny reference \\
        \tiny feedforward
      \end{tabular}} + \overbrace{m_e\mathbf{c}_d}^{\begin{tabular}{c}
        \tiny MPC \\
        \tiny feedforward
      \end{tabular}} + \overbrace{m_eg\mathbf{\hat{e}}_3}^{\begin{tabular}{c}
        \tiny gravity \\
        \tiny compensation
      \end{tabular}} + \\
      &\underbrace{-\mathbf{d}_w\circ \left[\begin{smallmatrix}
        1\\
        1\\
        0
      \end{smallmatrix}\right]}_{\begin{tabular}{c}
        \tiny world disturbance\\
        \tiny compensation
      \end{tabular}} + \underbrace{-\mathbf{d}_b\circ \left[\begin{smallmatrix}
        1\\
        1\\
        0
      \end{smallmatrix}\right]}_{\begin{tabular}{c}
        \tiny body disturbance\\
        \tiny compensation
      \end{tabular}}.
    \end{split}
  \end{equation}

  \scriptsize \textbf{MPC feedforward:} desired acceleration from a constrained linear MPC.
\end{block}

\end{columns}

\fullciteinbox{baca2021mrs}{}

\end{frame}

%%}

\subsection{State estimators}

%%{ Estimator

\begin{frame}

\frametitle{The MRS UAV System}

  \begin{block}{MRS UAV System block diagram}
    \begin{figure}
      \begin{adjustbox}{max totalsize={1.0\textwidth}{.65\textheight}, center}
        \input{./fig/tikz/pipeline_diagram_estimator.tex}
      \end{adjustbox}
    \end{figure}
  \end{block}

  \fullciteinbox{baca2021mrs}{}

\end{frame}

\begin{frame}

\frametitle{The MRS UAV System}

  \begin{columns}[c]

  \column{0.55\textwidth} % Left column and width

  \begin{block}{Bank of filters}

    \begin{figure}
      \begin{adjustbox}{max totalsize={1.0\textwidth}{.65\textheight}, center}
        \input{./fig/tikz/bank_of_filters.tex}
      \end{adjustbox}
    \end{figure}

  \end{block}

  \column{0.40\textwidth} % Right column and width

  \begin{itemize}
    \item simultaneous estimation of UAV state in multiple frames of reference
    \item automatic \& manual switching of the main estimator
    \item automatic detection of estimator failures
    \item switching of an estimator is synchronizes throughout the control pipeline
  \end{itemize}

  \end{columns}

  \fullciteinbox{petrlik2020robust}{}

\end{frame}

%%}

\subsection{Reference generation}

%%{ MPC Tracker

%% | ----------------------- MPC Tracker ---------------------- |

\begin{frame}
\frametitle{The MRS UAV System}

  \begin{block}{MRS UAV System block diagram}
    \begin{figure}
      \begin{adjustbox}{max totalsize={1.0\textwidth}{.65\textheight}, center}
        \input{./fig/tikz/pipeline_diagram_tracker.tex}
      \end{adjustbox}
    \end{figure}
  \end{block}

  \fullciteinbox{baca2021mrs}{}

\end{frame}

\begin{frame}
\frametitle{Model Predictive Control Tracker}

\textbf{Feedback and Feed-forward multicopter controllers benefit from a smooth and feasible control reference. A step in desired position or velocity is not a feasible reference.}

\begin{columns}[c]

\column{0.48\textwidth} % Left column and width

\begin{block}{Problem}
  \begin{itemize}
    \item real-time reference generation
    \item \SI{100}{\hertz} full state reference for controllers: $\dot{\mathbf{x}}, \ddot{\mathbf{x}}, \dot{\ddot{\mathbf{x}}}$, $\ddot{\ddot{\mathbf{x}}}$, where $\mathbf{x} = \begin{pmatrix}
    r_x, r_y, r_z, \eta
    \end{pmatrix}^\intercal$ (pose and heading)
    \item UAV dynamics constraints satisfaction
    \item user input: trajectory consisting of poses sampled at regular intervals
  \end{itemize}
\end{block}

\column{0.48\textwidth} % Right column and width

\begin{block}{Solution}
  \begin{itemize}
    \item onboard real-time simulation of the linear translational UAV dynamics
    \item linear MPC (LCQP) solved at 100 Hz
    \item states of the simulated model sampled and taken as control reference
    \item \SI{8}{\second} prediction horizon used for inter-UAV collision avoidance
  \end{itemize}
\end{block}

\end{columns}

\fullciteinbox{baca2018model}{}

\end{frame}

\begin{frame}
\frametitle{Model Predictive Control Tracker}

\includegraphics[width=1.0\textwidth,trim={0 0.5cm 0 1.0cm},clip]{./fig/photos/tracker.jpg}

\begin{block}{Properties}

  \vspace{-1em}

  \begin{columns}[c]

  \column{0.48\textwidth} % Left column and width

  \begin{itemize}
    \item Linear MPC is solved in real time
    \item Mutual collision avoidance prevent damage during experimentation
  \end{itemize}

  \column{0.48\textwidth} % Right column and width
  \begin{itemize}
    \item The LCQP can be solved reliably
    \item The tracker can handle unfeasible trajectory references from a user
  \end{itemize}

  \end{columns}

\end{block}

% \fullciteinbox{baca2018model}{http://github.com/ctu-mrs/mrs_uav_trackers}
\fullciteinbox{baca2018model}{}

\end{frame}

%%}

%%{ control performance showcase

\begin{frame}
\frametitle{Showcase of the capabilities}

\only<1>{
  \begin{block}{Aggressive trajectory following}
    \begin{center}
      \mymovie[autostart,loop,start=30]{
        \includegraphics[width=0.7\textwidth]{./fig/mrs_uav_system_thumbnail.jpg}
      }{./videos/mrs_uav_system.mp4}\\
      Video: \url{http://mrs.felk.cvut.cz/mrs-uav-system}
    \end{center}
  \end{block}
}

\only<2>{
  \begin{block}{Flip}
    \begin{center}
      \mymovie[autostart,loop]{
        \includegraphics[width=0.8\textwidth]{./fig/flip_thumbnail.jpg}
      }{./videos/flip.mp4}\\
    \end{center}
  \end{block}
}

\end{frame}

%%}

\subsection{Trajectory generation}

%%{ Trajectory generation

\begin{frame}
\frametitle{Trajectory Generation}

  \begin{block}{MRS UAV System block diagram}
    \begin{figure}
      \begin{adjustbox}{max totalsize={1.0\textwidth}{.65\textheight}, center}
        \input{./fig/tikz/pipeline_diagram_trajectory_generation.tex}
      \end{adjustbox}
    \end{figure}
  \end{block}

  \fullciteinbox{baca2021mrs}{}

\end{frame}

\begin{frame}
\frametitle{Trajectory Generation}

\begin{block}{Path $\rightarrow$ Feasible Trajectory}
\includegraphics[width=0.9\textwidth]{./fig/trajectory_generation.jpg}
\end{block}

\begin{center}
  Fork of {\color{blue} ethz-asl/mav\_trajectory\_generation}
\end{center}

\fullciteinbox{richter2016polynomial}{}
\fullciteinbox{burri2015realtime}{}

\end{frame}

\begin{frame}
\frametitle{Trajectory Generation --- Improvements and beyond}

\begin{itemize}
  \item fixed poorly-implemented lower bound segment time initialization
  \item fixed poorly-implemented trajectory sampling
  \item recursive segment sub-sectioning for meeting desired corridor constraints
  \item + smooth continuation of the prior UAV motion (Requires the \emph{MPC Tracker})
  \item + Fallback solution if the \texttt{QP} optimization fails
  \item + asynchronous execution and timeouting (with fallback solution)
  \item + result sanitization
\end{itemize}

\fullciteinbox{baca2021mrs}{}

\end{frame}

\begin{frame}
\frametitle{Trajectory Generation}

\begin{block}{DARPA SubT Virtual - 2017--2021}
  \begin{center}
    \mymovie[autostart,loop]{
      \includegraphics[width=0.8\textwidth]{./fig/darpa_virtual_cave_thumbnail.jpg}
    }{./videos/darpa_virtual_cave_short.mp4}\\
  \end{center}
\end{block}

\end{frame}

%%}

\subsection{Mapping and planning}

  %%{ Mapping Planning

  \begin{frame}
    \frametitle{Mapping \& Planning Pipeline}

    \begin{block}{Realtime Mapping \& Planning pipeline using LiDAR, RGBD}
      \begin{center}
        \mymovie[autostart,loop]{
          \includegraphics[width=0.8\textwidth]{fig/unreal_mapping_thumbnail.jpg}
        }{videos/unreal_mapping.mp4}\\
      \end{center}
    \end{block}

  \end{frame}

  %%}

%% | --------------------- Implementation --------------------- |

\section{MRS UAV System --- Implementation}

%%{ Our drones

\begin{frame}
  \frametitle{Our drones}

  \begin{columns}[c]

    \column{0.25\textwidth} % Left column and width

    \begin{figure}
      \centering
      \includegraphics[width=1.0\textwidth]{./fig/uavs/f330_real.jpg}
    \end{figure}

    \vspace{-1em}

    \begin{figure}
      \centering
      \includegraphics[width=1.0\textwidth]{./fig/uavs/f450_real.jpg}
    \end{figure}

    \vspace{-1em}

    \begin{figure}
      \centering
      \includegraphics[width=1.0\textwidth]{./fig/uavs/f550_real.jpg}
    \end{figure}

    \column{0.25\textwidth} % Right column and width

    \begin{figure}
      \centering
      \includegraphics[width=1.0\textwidth]{./fig/uavs/x500_real.jpg}
    \end{figure}

    \vspace{-1em}

    \begin{figure}
      \centering
      \includegraphics[width=1.0\textwidth]{./fig/uavs/t650_real.jpg}
    \end{figure}

    \vspace{-1em}

    \begin{figure}
      \centering
      \includegraphics[width=1.0\textwidth]{./fig/uavs/t690_real.jpg}
    \end{figure}

    \column{0.25\textwidth} % Right column and width

    \begin{figure}
      \centering
      \includegraphics[width=1.0\textwidth]{./fig/uavs/naki_real.jpg}
    \end{figure}

    \vspace{-1em}

    \begin{figure}
      \centering
      \includegraphics[width=1.0\textwidth]{./fig/uavs/eagle_real.jpg}
    \end{figure}

    \vspace{-1em}

    \begin{figure}
      \centering
      \includegraphics[width=1.0\textwidth]{./fig/uavs/dofec_real.jpg}
    \end{figure}

  \end{columns}

\end{frame}

%%}

%%{ What's on your tool belt?

\begin{frame}
  \frametitle{What's on your tool belt?}

  \begin{columns}[c]

    \column{0.48\textwidth} % Left column and width
    {\Huge Getting research done by:}
    \begin{itemize}
      \item using the tools provided by the community,
      \item adapting tools provided by the community,
      \item creating tools for yourself,
      \item providing tools for the community.
    \end{itemize}

    \vspace{2em}

    \only<2>{
      \begin{center}
        {\huge {\color{red} Automation}}
      \end{center}
    }

    \column{0.48\textwidth} % Right column and width
    \begin{figure}
      \centering
      \includegraphics[width=1.0\textwidth]{./fig/tool_belt.jpg}
    \end{figure}

  \end{columns}

\end{frame}

%%}

%%{ Gazebo

\section{Gazebo}

\begin{frame}

  \frametitle{Gazebo/ROS simulation pipeline}

  \begin{columns}[c]

    \column{0.5\textwidth} % Left column and width
    \begin{figure}
      \centering
      \includegraphics[width=0.5\textheight]{./fig/gazebo_logo.png}
    \end{figure}

    \column{0.5\textwidth} % Right column and width
    \only<2>{
      \begin{figure}
        \centering
        \includegraphics[width=0.8\textwidth]{./fig/leatherman.jpg}
      \end{figure}
    }

  \end{columns}

\end{frame}

%%}

%%{ Gazebo SITL

\subsection{Baseline UAV spawning}

\begin{frame}
  \frametitle{Gazebo/ROS SITL}

  \begin{figure}
    \centering
    \includegraphics[width=1.0\textwidth]{./fig/schematics/sitl.pdf}
  \end{figure}

\end{frame}

%%}

%%{ Gazebo Old spawning

\begin{frame}
  \frametitle{Gazebo/ROS PX4 traditional SITL spawning}

  \only<1>{
    \includegraphics[width=1.0\textwidth]{./fig/schematics/gazebo_old_1.pdf}
  }

  \only<2>{
    \includegraphics[width=1.0\textwidth]{./fig/schematics/gazebo_old_2.pdf}
  }

  \only<3>{
    \includegraphics[width=1.0\textwidth]{./fig/schematics/gazebo_old_3.pdf}
  }

\end{frame}

%%}

%%{ Sitl gazebo files

\begin{frame}
  \frametitle{Gazebo/ROS SITL Gazebo files}

  \begin{itemize}
    \item manual editing does not scale
  \end{itemize}

  \begin{figure}
    \centering
    \includegraphics[width=0.85\textwidth]{./fig/sitl_gazebo_files.png}
  \end{figure}

\end{frame}

%%}

%%{ Gazebo MRS spawning

\subsection{MRS custom spawning}

\begin{frame}
  \frametitle{Gazebo/ROS MRS PX4 SITL spawning}

  \begin{itemize}
    \item scalable for any UAV configuration
  \end{itemize}

  \only<1>{
    \includegraphics[width=1.0\textwidth]{./fig/schematics/gazebo_mrs_1.pdf}
  }

  \only<2>{
    \includegraphics[width=1.0\textwidth]{./fig/schematics/gazebo_mrs_2.pdf}
  }

  \only<3>{
    \includegraphics[width=1.0\textwidth]{./fig/schematics/gazebo_mrs_3.pdf}
  }

  \only<4>{
    \includegraphics[width=1.0\textwidth]{./fig/schematics/gazebo_mrs_4.pdf}
  }

  \only<5>{
    \includegraphics[width=1.0\textwidth]{./fig/schematics/gazebo_mrs_5.pdf}
  }

  \only<6>{
    \includegraphics[width=1.0\textwidth]{./fig/schematics/gazebo_mrs_6.pdf}
  }

\end{frame}

%%}

%%{ Customizing the UAV payload

\begin{frame}
  \frametitle{Parametric UAV-payload customization}

  \begin{columns}[c]

    \column{0.48\textwidth} % Left column and width

    \vspace{-1em}

    \onslide<1->{
      \begin{block}{1. Definition of universal payloads}
        \begin{itemize}
          \item Payload models with plugins defined in a separate file.
        \end{itemize}
      \end{block}
    }

    \onslide<2->{
      \vspace{-0.5em}

      \begin{block}{2. Definition of platform-specific mounting points}
        \begin{itemize}
          \item Arbitrary placement and configuration of particular payload for particular platform.
        \end{itemize}
      \end{block}
    }

    \onslide<3>{
      \vspace{-0.5em}

      \begin{block}{3. Dynamic assembly of the final UAV sdf structure}
        \begin{itemize}
          \item The robot description file is assembled based on user's query.
          \item Any-to-any mapping between UAV frames and UAV payloads
          \item Queries are queued and executed in series.
        \end{itemize}
      \end{block}
    }

    \column{0.48\textwidth} % Right column and width

    \vspace{-1em}

    \only<1>{
      \vspace{-2em}

      \begin{figure}
        \includegraphics[width=1.0\textwidth]{./fig/component_snippets.png}
      \end{figure}
    }

    \only<2>{
      \vspace{-2em}

      \begin{figure}
        \includegraphics[width=1.0\textwidth]{./fig/f450_sdf.png}
      \end{figure}
    }

    \only<3>{
      \begin{block}{Bonus facts}
        \begin{itemize}
          \item Potential definition of sensor groups.
          \item Support for query parameters --- additional customization.
        \end{itemize}
      \end{block}
    }

  \end{columns}

\end{frame}

%%}

%%{ How about sensor transformations?

\begin{frame}
  \frametitle{What about sensor static transformations?}

  \onslide<1->{
    \huge Static tfs are specific to each sensors individual placement on UAV!
  }

  \onslide<2->{
    \huge Only Gazebo knows them...
  }

  \vspace{1em}

  \onslide<3->{
    \begin{block}{\huge Solution:}
      \large World plugin that listens to Gazebo tfs and republishes them as ROS tfs.
    \end{block}
  }

\end{frame}

%%}

%%{ Gazebo screen

\begin{frame}
  \frametitle{Gazebo/ROS simulator}

  \begin{figure}
    \centering
    \includegraphics[width=0.9\textwidth]{./fig/gazebo_screen.jpg}
  \end{figure}

\end{frame}

%%}

%%{ Simulator UAVs

\begin{frame}
  \frametitle{Simulator UAVs}

  \begin{columns}[c]

    \column{0.48\textwidth} % Left column and width
    \includegraphics[width=1.0\textwidth]{./fig/simulator_uavs_1.jpg}

    \column{0.48\textwidth} % Right column and width
    \includegraphics[width=1.0\textwidth]{./fig/simulator_uavs_2.jpg}

  \end{columns}

\end{frame}

%%}

%%{ Live demo

\subsection{Spawner demo}

\begin{frame}
  \frametitle{Live demo!}

  \begin{center}
    \huge Live demo
  \end{center}

\end{frame}

%%}

  %%{ MRS UAV System --- Gazebo simulation

  \begin{frame}
    \frametitle{MRS UAV System --- Gazebo simulation}

    \vspace{-0.33em}

    \includegraphics[width=1.0\textwidth]{./fig/thumbnail_simulation.jpg}

    \vspace{-0.3em}

    \begin{itemize}
      \item 7 UAV types: DJI f330, DJI f450, DJI f550, T-Motor x500, Tarot 650, T-drone m690
    \end{itemize}

    \begin{columns}[c]

      \column{0.60\textwidth} % Left column and width

      \begin{block}{Available sensors}
        \begin{itemize}
          \item RGB camera: MatrixVision Bluefox, mobius
          \item Stereo Cameras: Intel Realsense
          \item 1D LiDARs: Terabee Teraranger, Gamin Lite
          \item 2D LiDARs: Scanse Sweep, Garmin RPLidar
          \item 3D LiDARs: Velodyne Puck, Ouster
          \item Thermal cameras, UV cameras for UVDAR
        \end{itemize}
      \end{block}

      \column{0.35\textwidth} % Right column and width

      \begin{block}{Available actuation}
        \begin{itemize}
          \item magnetic gripper
          \item parachute
          \item water gun
        \end{itemize}
      \end{block}

    \end{columns}

    \begin{center}
      \url{http://github.com/ctu-mrs/mrs_uav_gazebo_simulation}
    \end{center}

  \end{frame}

  \begin{frame}
    \frametitle{MRS UAV System --- Gazebo Simulation}

    \begin{block}{Realistic simulations of UAV grasping a brick}
      \begin{center}
        \mymovie[autostart,loop]{
          \includegraphics[width=0.8\textwidth]{fig/brick_grasping_simulated_thumbnail.jpg}
        }{videos/brick_grasping_simulated.mp4}\\
      \end{center}
    \end{block}

  \end{frame}

  \begin{frame}
    \frametitle{MRS UAV System --- Real world}

    \begin{block}{Real world UAV grasping a brick}
      \begin{center}
        \mymovie[autostart,loop]{
          \includegraphics[width=0.8\textwidth]{fig/brick_grasping_desert_thumbnail.jpg}
        }{videos/brick_grasping_desert.mp4}\\
      \end{center}
    \end{block}

  \end{frame}

  %%}

%%{ Sources

\begin{frame}
  \frametitle{How to get the MRS UAV System \& Sources}

  \begin{block}{Native installation above ROS Noetic}
    \texttt{curl https://ctu-mrs.github.io/ppa-stable/add\_ppa.sh | bash}\\
    \texttt{sudo apt install ros-noetic-mrs-uav-system-full}
  \end{block}

  \begin{block}{Apptainer container system + container wrapper}
    \begin{center}
      \large {\color{blue} http://github.com/ctu-mrs/mrs\_apptainer}
    \end{center}
  \end{block}

  \begin{block}{Docker containers}
    \begin{center}
      \large {\color{blue} http://github.com/ctu-mrs/mrs\_docker}
    \end{center}
  \end{block}

  \begin{block}{Sources}
    \begin{center}
      \large {\color{blue} http://github.com/ctu-mrs/mrs\_uav\_system}
    \end{center}
  \end{block}

  \begin{block}{How to}
    \large {\color{blue} https://ctu-mrs.github.io/docs/simulation/}
  \end{block}

\end{frame}

%%}

%%{ Is one simulator enough

\begin{frame}
  \frametitle{Is one simulator enough?}

  \only<1>{
    \begin{center}
      \huge Is one simulator enough?
    \end{center}
  }

  \only<2>{
    \begin{block}{Large-scale swarms --- require high parallelization within the simulator}
      \begin{center}
        \mymovie[autostart,loop]{
          \includegraphics[width=0.75\textwidth]{./fig/swarm_14uavs_field_thumbnail.jpg}
        }{./videos/swarm_14uavs_field.mp4}
      \end{center}
    \end{block}
  }

  \only<3>{
    \begin{block}{High-fidelity visuals --- requires capable rendering engine}
      \begin{center}
        \mymovie[autostart,loop]{
          \includegraphics[width=0.60\textwidth]{./fig/ue5_example_thumbnail.jpg}
        }{./videos/ue5_example.mp4}
      \end{center}
    \end{block}
  }

  % \only<4>{
  %   \begin{block}{CoppeliaSim --- Ease of use and deployment}
  %     \begin{center}
  %       \mymovie[autostart,loop]{
  %         \includegraphics[width=0.80\textwidth]{./fig/coppelia_thumbnail.jpg}
  %       }{./videos/coppelia.mp4}
  %     \end{center}
  %   \end{block}
  % }

  \only<5->{
    \begin{center}
      \huge Is one simulator enough?
      \huge {\color{red} No}\\
      \huge Each simulator has pros and cons:
    \end{center}
  }

  \only<6->{
    \begin{columns}[c]

      \column{0.3\textwidth} % Left column and width
      \only<6->{
        \begin{figure}
          \includegraphics[width=1.0\textwidth]{./fig/leatherman.jpg}
        \end{figure}
      }

      \column{0.3\textwidth} % Right column and width
      \only<7->{
        \begin{figure}
          \includegraphics[width=1.0\textwidth]{./fig/inflatable_tools.jpg}
        \end{figure}
      }

      \column{0.3\textwidth} % Right column and width

      \only<8->{
        \begin{figure}
          \includegraphics[width=1.0\textwidth]{./fig/screw_drivers.jpg}
        \end{figure}
      }

    \end{columns}
  }

\end{frame}

%%}

%%{ How to interact with simulators

\begin{frame}
  \frametitle{How to interact with simulators}

  \onslide<1->{
    \huge How to interact with various simulators?
  }

  \onslide<2>{
    \vspace{2em}

    \huge How to interact with various hardware of UAV platforms?
  }

\end{frame}

%%}

%%{ MRS UAV System

\begin{frame}
  \frametitle{MRS UAV System --- the ``old vendor-locked-in'' system architecture}

  \begin{figure}
    \centering
    \includegraphics[width=0.8\textwidth]{./fig/schematics/mrs_system_with_px4.pdf}
  \end{figure}

  \fullciteinbox{baca2021mrs}{}

\end{frame}

%%}

%%{ Various requirements and features

\begin{frame}
  \frametitle{Various requirements and functionalities of UAV platforms}

  \begin{columns}[c]

    \column{0.48\textwidth} % Left column and width
    \onslide<1->{
      \begin{block}{UAV Inputs}
        \begin{itemize}
          \item attitude rate + throttle,
          \item velocity + heading,
          \item individual actuators,
          \item etc.
        \end{itemize}
      \end{block}
    }

    \column{0.48\textwidth} % Right column and width
    \onslide<2>{
      \begin{block}{Provided data}
        \begin{itemize}
          \item IMU,
          \item GNSS,
          \item attitude,
          \item velocity,
        \end{itemize}
      \end{block}
    }

  \end{columns}

\end{frame}

%%}

%%{ Solution

\begin{frame}
  \frametitle{Solutions?}

  \onslide<2->{
    \begin{block}{Naive and the simplest solution at the beginning}
      \begin{itemize}
        \item Adding options and different configurations of your control system for each platform API.
        \item {\color{green} low overhead in the beginning}
        \item {\color{red} not scalable, hard-to-maintain, ties the system with dependencies}
      \end{itemize}
    \end{block}
  }

  \onslide<3>{
    \begin{block}{Better solution for the long run}
      \begin{itemize}
        \item Add an \textbf{abstraction} layer
        \item {\color{green} scales well, removes dependencies, provides isolation from particular technology}
        \item {\color{red} introduces implementation and performance overhead}
      \end{itemize}
    \end{block}
  }
\end{frame}

%%}

%%{ mrs uav sytem HW api

\begin{frame}
  \frametitle{MRS UAV System --- hardware API}

  \only<1>{
    \begin{figure}
      \centering
      \includegraphics[width=1.0\textwidth]{./fig/schematics/hw_api_1.pdf}
    \end{figure}
  }

  \only<2>{
    \begin{figure}
      \centering
      \includegraphics[width=1.0\textheight]{./fig/schematics/hw_api_2.pdf}
    \end{figure}
  }

  \only<3>{
    \begin{figure}
      \centering
      \includegraphics[width=1.0\textwidth]{./fig/schematics/hw_api_3.pdf}
    \end{figure}
  }

  \only<4>{
    \begin{figure}
      \centering
      \includegraphics[width=1.0\textwidth]{./fig/schematics/hw_api_4.pdf}
    \end{figure}
  }

\end{frame}

%%}

%%{ plugin api

\subsection{General flight controller}

\begin{frame}
  \frametitle{Plugin API --- Definition of a general flight controller}

  \begin{columns}[c]

    \column{0.48\textwidth} % Left column and width
    \begin{block}{Information provided}
      Subset of the following:
      \begin{itemize}
        \item IMU
        \item GNSS
        \item RTK GNSS
        \item Magnetometer
        \item AMSL measurement
        \item Ground truth pose
        \item Height measurement
        \item Angular rate
        \item RC channels
        \item Velocity
        \item Orientation
        \item 3D Pose
      \end{itemize}
    \end{block}

    \column{0.48\textwidth} % Right column and width
    \begin{block}{Control input accepted}
      Subset of the following:
      \begin{itemize}
        \item Position \& heading
        \item Velocity \& heading
        \item Velocity \& heading rate
        \item Acceleration \& heading
        \item Acceleration \& heading rate
        \item Attitude \& throttle
        \item Attitude rate \& throttle
        \item Control group \& throttle
        \item Individual actuators' throttle
      \end{itemize}
    \end{block}

  \end{columns}

\end{frame}

%%}

%%{ What other simulators do we use?

\begin{frame}
  \frametitle{What other simulators?}

  \begin{columns}[c]

    \column{0.33\textwidth} % Left column and width
    \begin{block}{Coppelia}
      \begin{itemize}
        \item Easy to use
        \item Simple dependencies
        \item {\color{red} bad multi-uav performance}
      \end{itemize}
    \end{block}

    \column{0.65\textwidth} % Right column and width
    \begin{center}
      \mymovie[autostart,loop]{
        \includegraphics[width=1.00\textwidth]{./fig/coppelia_thumbnail.jpg}
      }{./videos/coppelia.mp4}
    \end{center}

  \end{columns}

\end{frame}

%%}

%%{ Custom MRS simulator

\begin{frame}
  \frametitle{Custom dynamics simulation --- github.com/ctu-mrs/mrs\_multirotor\_simulator}

  \vspace{-1em}

  \begin{columns}[c]

    \column{0.48\textwidth} % Left column and width

    \begin{block}{MRS multirotor simulator}
      \begin{itemize}
        \item Full multirotor dynamics
        \item Embedded feedback controllers
        \item Fast C\texttt{++} ODE solver
        \item \textbf{header-only library --- {\color{red} intended for RL}}
        \item available ROS integration
        \item minimum external dependencies
        \item tightly integrated into the MRS system's core
      \end{itemize}
    \end{block}

    \column{0.48\textwidth} % Right column and width

    \begin{block}{Available control inputs}
      \small
      \begin{itemize}
        \item Position \& heading
        \item Velocity \& heading
        \item Velocity \& heading rate
        \item Acceleration \& heading
        \item Acceleration \& heading rate
        \item Attitude \& throttle
        \item Attitude rate \& throttle
        \item Control group \& throttle
        \item Individual actuators' throttle
      \end{itemize}
    \end{block}
  \end{columns}

  \begin{figure}
    \includegraphics[width=1.0\textwidth]{./fig/schematics/full_dynamics.png}
  \end{figure}

\end{frame}

%%}

%%{ Live demo

\begin{frame}
  \frametitle{The MRS simulator --- swarm of 400 UAVs}

  \begin{block}{The MRS swarm simulator}
    \begin{center}
      \mymovie[autostart,loop]{
        \includegraphics[width=0.75\textwidth]{./fig/400_swarm_thumbnail.jpg}
      }{./videos/400_swarm.mp4}\\
    Video: \url{https://youtu.be/2UJ7aYaHOX0}
    \end{center}
  \end{block}

\end{frame}

%%}

%%{ Unreal Engine Drone Simulator

\begin{frame}
  \frametitle{Unreal Engine Drone Simulator (WIP)}

  \begin{columns}[c]

    \column{0.21\textwidth} % Left column and width

    \begin{itemize}
      \item Work-in-progress
      \item RGB camera
      \item Stereo cam
      \item 3D LiDAR
      \item Step-locked with our dynamics simulator (ROS)
    \end{itemize}

    \column{0.78\textwidth} % Right column and width

    \begin{center}
      \mymovie[autostart,loop]{
        \includegraphics[width=0.80\textwidth]{./fig/unreal_showcase_thumbnail.jpg}
      }{./videos/unreal_showcase.mp4}
    \end{center}

  \end{columns}

  \begin{center}
    \large \url{https://github.com/ctu-mrs/mrs_uav_unreal_simulation}
  \end{center}

\end{frame}

%%}

%%{ MRS UAV System

\begin{frame}
  \frametitle{MRS UAV System --- all open-source}

  \begin{figure}
    \centering
    \includegraphics[width=0.7\textwidth]{./fig/github.png}
    \huge {\color{blue} http://github.com/ctu-mrs}
  \end{figure}

\end{frame}

%%}

  %%{ MRS wiki

  \begin{frame}
    \frametitle{MRS UAV System wiki}
    \begin{figure}
      \vspace{-1em}
      \caption*{\url{http://ctu-mrs.github.io}}
      \includegraphics[width=0.7\textwidth]{fig/wiki.png}
    \end{figure}

  \end{frame}

  %%}

  %%{ MRS Cheat Sheet

  % \begin{frame}
  %   \frametitle{The MRS Cheat Sheet}
  %   \vspace{-1.0em}
  %   \begin{figure}
  %     \caption*{\url{http://github.com/ctu-mrs/mrs_cheatsheet}}
  %     \vspace{-0.5em}
  %     \includegraphics[width=0.8\textwidth]{fig/mrs_cheatsheet.png}
  %   \end{figure}
  % \end{frame}

  %%}

%%{ Conclusion

\begin{frame}
  \frametitle{Conclusions}

  \begin{center}
    \huge Thanks for your attention\\
  \end{center}

  \begin{block}{Download pdf of this presentation}
    \centering
    \Large\url{https://github.com/ctu-mrs/presentation_mrs_uav_system}
  \end{block}

  \begin{block}{My profile}
    \centering
    \Large\url{http://mrs.felk.cvut.cz/people/tomas-baca}
  \end{block}

  \begin{block}{\centering \textbf{Cite the MRS UAV System if you use it for your research}}
    \small \fullcite{baca2021mrs}
  \end{block}

\end{frame}

%%}

%%{ References

\DeclareCiteCommand{\fullcite}
{\usebibmacro{prenote}}
{\clearfield{addendum}%
  \usedriver
  {\defcounter{minnames}{6}%
  \defcounter{maxnames}{6}}
{\thefield{entrytype}}}
{\multicitedelim}
{\usebibmacro{postnote}}

\begin{frame}[allowframebreaks]
  \frametitle{References}
  \tiny{
    \printbibliography[heading=none,title={}]
  }
\end{frame}

%%}

\end{document}
