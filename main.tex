%----------------------------------------------------------------------------------------
%	PACKAGES AND THEMES
%----------------------------------------------------------------------------------------

\documentclass[aspectratio=169]{beamer}

\mode<presentation> {

  \usetheme{Boadilla} % light

  \usecolortheme{seahorse} % light

  \input{./cvut_colors.tex}

  % \setbeamertemplate{footline} % remove the footer line
  % \setbeamertemplate{footline}[page number] % replace the footer line with simple numbers

  \setbeamertemplate{navigation symbols}{}
  \setbeamertemplate{bibliography item}{\insertbiblabel} % removing the navigation symbols

}

%%{ Docu HEAD

\usepackage{graphicx} % Allows including images
\usepackage{booktabs} % Allows the use of \toprule, \midrule and \bottomrule in tables
\usepackage{multimedia}
\newcommand{\superfill}{\vskip0pt plus 1filll}

\usepackage{isotope}
\usepackage{animate}

\usepackage[export]{adjustbox}

\usepackage{graphicx}
\usepackage{setspace}
\usepackage{epstopdf}
\usepackage{float}
\usepackage{multirow,tabularx,makecell}

\usepackage{pdfpcnotes}

\usefonttheme{professionalfonts}
\usepackage{amsmath,amsfonts,amssymb,bm}

\usepackage[backend=bibtex,defernumbers=true,style=ieee,sorting=none,sortcites=false]{biblatex}

\renewcommand*{\bibfont}{\normalfont\tiny}

% Print labelnumbers with suffixes, adjust secondary labelnumber 2/2
\DeclareFieldFormat{labelnumber}{%
  \ifkeyword{mine}
  {\ifkeyword{core}
  {{\number\numexpr#1}}%
  {{\number\numexpr#1}}%
  }%
  {#1}%
  }

  \DeclareCiteCommand{\tabcite}%[\mkbibbrackets]
  {\usebibmacro{cite:init}%
  \usebibmacro{prenote}}
  {\usebibmacro{citeindex}%
  \usebibmacro{cite:comp}}
  {}
  {\usebibmacro{cite:dump}%
  \usebibmacro{postnote}}

  % {{\number\numexpr#1-\value{bbx:primcount}}a}

  %%{ fullcite box

  \usepackage{tcolorbox}

  \definecolor{light-gray}{gray}{0.95}
  \newcommand{\fullciteinbox}[2]{%

    \DeclareCiteCommand{\fullcite}
    {\usebibmacro{prenote}}
    {\clearfield{addendum}%
    \clearfield{issn}%
    \usedriver
    {\defcounter{minnames}{6}%
    \defcounter{maxnames}{6}}
    {\thefield{entrytype}}}
    {\multicitedelim}
    {\usebibmacro{postnote}}

    %\vspace{3em}%
    %\raisebox{3em}[3em][3em]{%
    % \vspace{-0.2em}
    % \begin{tcolorbox}[width=\textwidth,colback={light-gray},title={}]%
    \begin{block}{}
      \begin{minipage}[t]{0.07\linewidth}%
        \raggedright%
        \scriptsize \cite{#1}%
      \end{minipage}%
      \begin{minipage}[t]{0.93\linewidth}%
        \scriptsize \fullcite{#1}%
        \ifx&#2&
        \else
        \\
        \url{#2}
        \fi
      \end{minipage}%
      % \end{tcolorbox}%
    \end{block}
    %}%
    \vspace{-0.3em}
    }%

    %%}

    \addbibresource{main.bib}

    \defbibenvironment{favoritebib}
  {\itemize}
    {\enditemize}
  {\item}
    \usepackage{siunitx}
    \DeclareSIUnit \parsec {pc}
    \DeclareSIUnit \electronvolt {eV}
    \DeclareSIUnit \pixel {px}
    \DeclareSIUnit \arcmin {arcmin}
    \DeclareSIUnit \erg {erg}
    \DeclareSIUnit \joul {J}
    \DeclareSIUnit \Bq {Bq}

    \usepackage{enumitem} % To enable enumerate with letters (a, b, c...)
    \newlist{alphalist}{enumerate}{1}
    \setlist[alphalist]{label=\alph*)}
    \setlist[itemize]{label=\textbullet}

    \usepackage{cellspace}
    \newcolumntype{D}{>{\hfill}N{3}{2}<{\hfill}}
    \newcommand*\cellspacelimit[2]{\setlength{\cellspacetoplimit}{#1}\setlength{\cellspacebottomlimit}{#2}}

    % figures
    \usepackage{wrapfig}
    % \usepackage[font={footnotesize}]{caption}
    \usepackage[font={small}]{caption}
    % \usepackage{subcaption}

    % subfloat
    \usepackage{subfig}
    % \usepackage[export]{adjustbox}

    \usepackage{color}
    \usepackage{url}

    %%{ tikz

    \usepackage{tikz}
    \usepackage{pgfplots}
    \pgfplotsset{compat=1.14}
    \usetikzlibrary{backgrounds,arrows,automata,shapes,positioning,calc,through,spy,shapes,shapes.geometric,shapes.multipart,fit,patterns,fadings}
    \pgfdeclarelayer{background}
    \pgfdeclarelayer{foreground}
    \pgfsetlayers{background,main,foreground}

    \input{./fig/tikz/tikz.tex}

    \tikzset{
      imgletter/.style={
        rectangle,
        inner sep=2pt,
        rounded corners=.1em,
        text=black,
        minimum height=1em,
        text centered,
        fill=white,
        fill opacity=1.0,
        text opacity=1,
        anchor=south west,
      },
    }

    %%}

    \usepackage{pdfpc-movie}
    \newcommand{\mymovie}[3][]{\pdfpcmovie[#1]{#2}{#3}}
    % \newcommand{\mymovie}[3][]{\movie[#1]{#2}{#3}}

    %%{ ARROWS IN TIKZ

    \tikzset{
      myarrow/.style={
        draw,
        fill=orange,
        single arrow,
        minimum height=3.5ex,
        single arrow head extend=1ex
      }
    }

    \newcommand{\arrowup}{%
      \vspace{-0.8em}
      \tikz [baseline=-0.5ex]{\node [myarrow,rotate=90] {};}
      \vspace{-1.4em}
      }

      \newcommand{\arrowdown}{%
        \vspace{-0.8em}
        \tikz [baseline=-1ex]{\node [myarrow,rotate=-90] {};}
        \vspace{-1.5em}
        }

        \newcommand{\arrowright}{%
          \tikz [baseline=-1ex]{\node [myarrow,rotate=0] {};}
          }

          \newcommand{\arrowleft}{%
            \tikz [baseline=-1ex]{\node [myarrow,rotate=180] {};}
            }

            %%}

            %%{ CHECKMARK IN TIKZ

            \def\checkmark{\tikz\fill[scale=0.4](0,.35) -- (.25,0) -- (1,.7) -- (.25,.15) -- cycle;}

            %%}

    \newcommand{\strong}[1]{\textbf{#1}}
    \newcommand{\coord}[1]{\textbf{#1}}
    \newcommand{\norm}[1]{\left\lvert#1\right\rvert}
    \newcommand{\m}[1]{\ensuremath{\mathbf{#1}}}
    \newcommand\numberthis{\addtocounter{equation}{1}\tag{\theequation}}
    \newcommand{\corrected}[1]{{\color{black} {#1}}}
    % \newcommand{\comment}[1]{{\color{blue} {#1}}}
    \newcommand{\add}[1]{{\color{green} {#1}}}
    \newcommand{\todo}[1]{{\color{red} TODO {#1}}}
    \newcommand{\updated}[1]{{\color{blue} {#1}}}
    \newcommand{\reffig}[1]{Fig.~\ref{#1}}
    \newcommand{\refalg}[1]{Alg.~\ref{#1}}
    \newcommand{\refsec}[1]{Sec.~\ref{#1}}
    \newcommand{\reftab}[1]{Table~\ref{#1}}
    \newcommand{\real}{\mathbb{R}}
    \newcommand{\red}[1]{{\color{red} #1}}
    \newcommand{\minus}{\scalebox{0.75}[1.0]{$-$}}
    \newcommand{\plus}{\scalebox{0.8}[0.8]{$+$}}
    \newcommand{\figvspace}{\vspace{-1em}} % this may eventually do something, so far just a placeholder

    % \usepackage{pgf}
    % \logo{\pgfputat{\pgfxy(0,5)}{\pgfbox[right,base]{\includegraphics[height=0.8cm]{}}}}
    % \newcommand{\nologo}{\setbeamertemplate{logo}{}}

    % \usepackage{eso-pic}
    % \newcommand\AtPagemyUpperLeft[1]{\AtPageLowerLeft{\put(\LenToUnit{0.66\paperwidth},\LenToUnit{0.904\paperheight}){#1}}}
    % \AddToShipoutPictureFG{
    %   \AtPagemyUpperLeft{{\includegraphics[height=0.85cm,keepaspectratio]{fig/logo_ctu_fee_mrs_blue.png}}}
    % }
    % \newcommand{\AddToShipoutPictureFG}{\setbeamertemplate{logo}{}}

    %%}

    %%{ TITLE PAGE


    \title[MRS UAV System]{MRS UAV System for real-world testing and development}
    \subtitle{From control theory to practice}
    \author[Tomas Baca]{{Tomas Baca, Ph.D.} \\
    }
    \institute[CTU in Prague] {
      \begin{small}
        Multi-Robot Systems group, Faculty of Electrical Engineering\\
        Czech Technical University in Prague
      \end{small}}

    \date[August 1st, 2022]{}

    \titlegraphic{\includegraphics[width=5cm]{fig/logo_ctu_fee_mrs_blue.png}}

    \begin{document}

    \begin{frame}

      \titlepage % Print the title page as the first slide

    \end{frame}

    % \nologo

    %%}

\begin{frame}
  \frametitle{Outline}
  \tableofcontents
\end{frame}

\section{UAV Experimentation in MRS}

%%{ kumar

\begin{frame}
\frametitle{Motivation --- Awesome multi-UAV videos}

\begin{block}{2012, University of Pennsylvania, prof. Vijay Kumar}
  \begin{center}
    \mymovie[autostart,loop]{
      \includegraphics[width=0.7\textwidth]{./fig/kumar_thumbnail.jpg}
    }{./videos/kumar.mp4}\\
    Video: \url{https://youtu.be/4ErEBkj_3PY}
  \end{center}
\end{block}

\end{frame}

%%}

%%{ UAV Experimentation in MRS

\begin{frame}
  \frametitle{UAV Experimentation in MRS --- 2012--2015}

  \only<+>{\begin{block}{$\approx$ 2012, designing custom PCBs for controllers}
    \centering
    \includegraphics[width=0.48\textwidth]{fig/photos/pcb1.jpg}
    \includegraphics[width=0.48\textwidth]{fig/photos/first_UAV.jpg}
  \end{block}}

  \only<+>{\begin{block}{$\approx$ 2014, Model Predictive Control on embedded hardware}
    \centering
    \includegraphics[width=0.48\textwidth]{fig/photos/pcb2.jpg}
    \includegraphics[width=0.48\textwidth]{fig/photos/prase.jpg}

    \fullciteinbox{baca2016embedded}{}
  \end{block}}

  \only<+>{\begin{block}{Embedded Model Predictive Control}
    \begin{center}

      \mymovie[autostart,loop]{
        \includegraphics[width=0.7\textwidth]{fig/photos/embedded_mpc_thumbnail.jpg}
      }{videos/embedded_mpc.mp4}\\
      Video: \url{http://youtu.be/AXI_rkQRBaE}

    \end{center}
  \end{block}}

  \only<+>{\begin{block}{Embedded Model Predictive Control}
    \begin{center}
      \mymovie[autostart,loop]{
        \includegraphics[width=0.7\textwidth]{fig/photos/mmar_vojta_thumbnail.jpg}
      }{videos/mmar_vojta.mp4}\\
      Video: \url{http://youtu.be/9Bpm4J31CgE}
    \end{center}
  \end{block}}

\end{frame}

\begin{frame}
  \frametitle{UAV Experimentation in MRS --- 2012--2015}

  \begin{itemize}
    \item Custom and purpose-built hardware
    \item Embedded software
    {\color{red} \item not Scalable! Bottlenecks everywhere
    \item embedded programming is \emph{cumbersome}
    \item simulating is difficult (custom-built hardware does not help)
    \item system is hard-to-maintain
    \item low modularity
    \item available sensors and peripheries limited}
  \end{itemize}

  \begin{block}{Publications}
    \cite{saska2013adhoc}, \cite{chudoba2014localization}, \cite{baca2016embedded}, \cite{saska2017documentation}, \cite{chudoba2016exploration}, \cite{saska2016formations}, \cite{spurny2016complex}, \cite{saska2017system}
  \end{block}

\end{frame}

%%}

%%{ CURRENT SYSTEM diagram

\begin{frame}
\frametitle{Current System Diagram}

\begin{center}
  \includegraphics[width=0.8\textwidth]{./fig/new_diagram.png}
\end{center}

\end{frame}

%%}

            %%{ ROS -- Robot Operating System

            \begin{frame}
              \frametitle{ROS -- Robot Operating System}

              \begin{itemize}
                \item middleware allowing communication between programs
                \item integrates with \texttt{C++}, Python, Bash and Zshell
                \item makes the transition from \emph{Matlab} to reality bearable
                \item supported by sensor manufacturers
                \item integration through the Linux terminal
                \item out of the box: time and clock management, logging, recording onboard data, visualization and plotting, parameter loading, static and dynamic transformations, etc.
                \item integrates to robotic simulators: Gazebo, Coppelia (V-REP)
              \end{itemize}

              \begin{figure}
                \includegraphics[width=0.3\textwidth]{fig/ros_logo.jpg}
              \end{figure}

            \end{frame}

            %%}

%% | ------------ Control pipeline - implementation ----------- |

\section{UAV Control pipeline}
\subsection{Control -- math point of view}

%%{ UAV control pipepline

\begin{frame}
  \frametitle{Control theory point of view}

  \begin{figure}
    \begin{adjustbox}{max totalsize={1.0\textwidth}{.85\textheight}, center}
      \input{fig/tikz/pipeline_diagram.tex}
    \end{adjustbox}
  \end{figure}

  \fullciteinbox{baca2021mrs}{}

\end{frame}

%%}

\subsection{Control -- implementation point of view}

%%{ From theory to practical experiments

%%{ Linear Kalman filter

\begin{frame}
  \frametitle{From theory to practical experiments}

  \begin{columns}[c]

    \onslide<1->{\column{0.48\textwidth} % Left column and width
    \begin{block}{Linear Kalman Filter $\approx$ 10 lines}
      \begin{adjustbox}{max totalsize={1.0\textwidth}{.85\textheight}, center}
        \input{fig/tikz/kalman.tex}
      \end{adjustbox}
    \end{block}}

    \onslide<2>{\column{0.48\textwidth} % Right column and width
    \begin{block}{Linear Kalman Filter in practice}
      \begin{itemize}
        \item $\ge$ 1000 lines of \texttt{C++} code
        \item Real-world fusion \& estimation\\
          $\ge$ 10 000 lines of \texttt{C++} code
      \end{itemize}
    \end{block}}

  \end{columns}

\end{frame}

%%}

%%{ Control

\begin{frame}
  \frametitle{From theory to practical experiments}

  \begin{columns}[c]

    \column{0.60\textwidth} % Left column and width
    \onslide<1->{\vspace{-1em}
\begin{block}{Geometric tracking controller $\approx$ 10 lines}
\vspace{-1em}
\tiny\begin{align}
\mathbf{f}_d &= \minus m_e\mathbf{k}_p\circ \mathbf{e}_p \minus m_e\mathbf{k}_v\circ \mathbf{e}_v \plus m_e\ddot{\mathbf{r}}_d \minus m_eg\mathbf{\hat{e}}_3 \minus \mathbf{d}_w\circ \left[\begin{smallmatrix}
1\\
1\\
0
\end{smallmatrix}\right] \minus \mathbf{d}_b\circ \left[\begin{smallmatrix}
1\\
1\\
0
\end{smallmatrix}\right],\nonumber\\
\mathbf{e}_R &= \frac{1}{2}\left(\mathbf{R}_d^\intercal\mathbf{R} - \mathbf{R}^\intercal\mathbf{R}_d\right),\nonumber\\
\bm{\omega}_d &= \minus \mathbf{k}_R\circ \mathbf{e}_R \plus \bm{\omega}_j \minus \bm{\omega}_c,\nonumber
\end{align}
\vspace{-1em}
\end{block}}

\onslide<2->{
\vspace{-0.5em}
\begin{block}{MPC controller $\approx$ 100 lines}
\vspace{-1em}
\scriptsize{
\begin{align*}
  & \min_{\mathbf{u}_{[1:n]}}\nonumber
  & \frac{1}{2}\sum_{i=1}^{n-1}\left(\mathbf{e}^\intercal_{[i]}\mathbf{Q}\mathbf{e}_{[i]}\right) + \mathbf{e}^\intercal_{[n]}\mathbf{S}\mathbf{e}_{[n]}
\end{align*}\begin{align*}
  \text{s.t.}~ \mathbf{x}_{m[i]} &= \mathbf{A}_m\mathbf{x}_{m[i-1]} \plus \mathbf{B}_m\mathbf{u}_{[i]}, &\forall i &\in \{1, \hdots, n\}\nonumber\\
  \mathbf{x}_{m[i]} &\leq \mathbf{x}_{\mathrm{max}}, &\forall i &\in \{1, \hdots, n\}\nonumber\\
  \mathbf{x}_{m[i]} &\geq \minus \mathbf{x}_{\mathrm{max}}, &\forall i &\in \{1, \hdots, n\}\nonumber \\
  \mathbf{u}_{[i]} \minus \mathbf{u}_{[i\minus 1]} &\leq \dot{\mathbf{u}}_{\mathrm{max}} \Delta t, &\forall i &\in \{2, \hdots, n\}\nonumber\\
  \mathbf{u}_{[i]} \minus \mathbf{u}_{[i\minus 1]} &\geq \minus \dot{\mathbf{u}}_{\mathrm{max}} \Delta t, &\forall i &\in \{2, \hdots, n\}\nonumber
\end{align*}
}
\vspace{-1em}
\end{block}}

\column{0.38\textwidth} % Right column and width

\onslide<3>{\begin{block}{UAV Control in practice}
\begin{itemize}
\item $\ge$ 10 000 lines of \texttt{C++} code
\end{itemize}
\end{block}}

\end{columns}

\end{frame}

%%}

%%{ From theory to practical experiments

\begin{frame}
  \frametitle{From theory to practical experiments}

  \begin{columns}[c]

    \column{0.70\textwidth} % Left column and width
    \begin{block}{What needs to be solved outside of Matlab's sandbox?}
      \begin{itemize}
        \item UAV crashes are expensive, so don't crash
        \item Software runtime errors cause UAV crashes
        \item control references might not be feasible
        \item sensors can get disconnected during the flight
        \item takeoff and landing: \textbf{the most tricky part of the flight}
        \item mass (thus the model) can change during the flight
        \item controllers can be poorly tuned... handle instabilities
        \item acceleration and speed depend on the available sensors
        \item people are fallible, don't let them crash the drones
        \item not all states of UAV are allowed, even though controllers can reach them (upside down)
      \end{itemize}
    \end{block}

    \column{0.28\textwidth} % Right column and width
    \begin{block}{The failsafe core}
      \texttt{if (goingToCrash())}\\
      \texttt{~~~~dont();}
    \end{block}

  \end{columns}

\end{frame}

%%}

\begin{frame}
\frametitle{Pushing the chances --- 14 UAVs in a swarm}

  \begin{center}
    \mymovie[autostart,loop]{
      \includegraphics[width=0.9\textwidth]{./fig/swarm_14uavs_field_thumbnail.jpg}
    }{./videos/swarm_14uavs_field.mp4}\\
  \end{center}

\end{frame}

\begin{frame}
\frametitle{Pushing the chances too much}

\begin{center}
  \mymovie[autostart,loop]{
    \includegraphics[width=0.9\textwidth]{./fig/crashes_thumbnail.jpg}
  }{./videos/crashes.mp4}\\
\end{center}

\end{frame}

%%}

%% | --------------------- MRS UAV System --------------------- |

%%{ System diagram

\begin{frame}

\frametitle{The MRS UAV System}

  \begin{block}{MRS UAV System block diagram}
    \begin{figure}
      \begin{adjustbox}{max totalsize={1.0\textwidth}{.65\textheight}, center}
        \only<1>{\input{./fig/tikz/pipeline_diagram.tex}}
        \only<2>{\input{./fig/tikz/pipeline_diagram_controller.tex}}
      \end{adjustbox}
    \end{figure}
  \end{block}

  \fullciteinbox{baca2021mrs}{}

\end{frame}

%%}

\subsection{Reference controllers}

%%{ Reference controllers

%%{ Heading-compliant control design

\begin{frame}
\frametitle{Reference controllers --- Attitude rate}

\begin{block}{\small Heading-compliant control design}
  \small Heading $\eta = \mathrm{atan2}\left(\mathbf{\hat{b}}_1^\intercal\mathbf{\hat{e}}_2, \mathbf{\hat{b}}_1^\intercal\mathbf{\hat{e}}_1\right) = \mathrm{atan2}\left(\mathbf{h}_{(2)}, \mathbf{h}_{(1)}\right)$ as the $4^{\mathrm{th}}$ DOF (instead of \emph{yaw)}.
\end{block}

\begin{columns}[c]

\column{0.48\textwidth} % Left column and width

\begin{block}{\small \emph{SO(3)} attitude controller \cite{lee2010geometric}}
  \begin{itemize}
    \item \scriptsize Custom heading-compliant desired orientation:
  \end{itemize}

  \begin{equation}
    \scriptsize
    \mathbf{R}_d = \left[\mathbf{\hat{b}}_{1d}, \mathbf{\hat{b}}_{2d}, \mathbf{\hat{b}}_{3d}\right],
  \end{equation}

  \begin{equation}
    \scriptsize
    \mathbf{b}_{1d} = \mathbf{O}(\mathbf{P}^\intercal\mathbf{O})^{-1}\mathbf{P}^\intercal\,\mathbf{\hat{h}}_d,\ \mathbf{\hat{b}}_{1d} = \frac{\mathbf{b}_{1d}}{\|\mathbf{b}_{1d}\|},
  \end{equation}

  \begin{equation}
    \scriptsize
    \hat{\mathbf{b}}_{2d} = \mathbf{\hat{b}}_{3d}\times \mathbf{\hat{b}}_{1d}.
  \end{equation}

\end{block}

\column{0.48\textwidth} % Right column and width

  \vspace{-1em}

  \begin{figure}
    \centering
    \includegraphics[width=0.90\textwidth]{fig/sketch/coordinate_frames_platform.pdf}
    % \caption{
    %   The image depicts the world frame $\mathcal{W}$ = $\{\mathbf{\hat{e}}_1$, $\mathbf{\hat{e}}_2$, $\mathbf{\hat{e}}_3\}$ in which the 3D position and the orientation of the \ac{UAV} body is expressed.
    %   The body frame $\mathcal{B}$ = $\{\mathbf{\hat{b}}_1$, $\mathbf{\hat{b}}_2$, $\mathbf{\hat{b}}_3\}$ relates to $\mathcal{W}$ by the translation $\mathbf{r} = \left[x, y, z\right]^{\intercal}$ and by rotation $\mathbf{R}^{\intercal}$.
    %   The \ac{UAV} heading vector $\mathbf{h}$, which is a projection of $\hat{\mathbf{b}}_1$ to the plane $span\left(\mathbf{\hat{e}}_1, \mathbf{\hat{e}}_2\right)$, forms the heading angle $\eta = \mathrm{atan2}\left(\mathbf{\hat{b}}_1^\intercal\mathbf{\hat{e}}_2, \mathbf{\hat{b}}_1^\intercal\mathbf{\hat{e}}_1\right) = \mathrm{atan2}\left(\mathbf{h}_{(2)}, \mathbf{h}_{(1)}\right)$.
    %   }
  \end{figure}

\end{columns}

\fullciteinbox{baca2021mrs}{}

\end{frame}

%%}

\begin{frame}
\frametitle{Reference controllers --- Desired force and heading rate}

\begin{columns}[c]

\column{0.48\textwidth} % Left column and width

\begin{block}{\small Geometric tracking on \emph{SE(3)} \cite{lee2010geometric}}
  \begin{equation}
    \scriptsize
    \begin{split}
      \mathbf{f}_d = &\overbrace{-m_e\mathbf{k}_p\circ \mathbf{e}_p}^{\begin{tabular}{c}
        \tiny position\\
        \tiny feedback
      \end{tabular}} + \overbrace{-m_e\mathbf{k}_v\circ \mathbf{e}_v}^{\begin{tabular}{c}
        \tiny velocity\\
        \tiny feedback
      \end{tabular}} + \overbrace{m_e\ddot{\mathbf{r}}_d}^{\begin{tabular}{c}
        \tiny reference\\
        \tiny feedforward
      \end{tabular}} + \\
      & \underbrace{m_eg\mathbf{\hat{e}}_3}_{\begin{tabular}{c}
        \tiny gravity\\
        \tiny compensation
      \end{tabular}} + \underbrace{-\mathbf{d}_w\circ \left[\begin{smallmatrix}
        1\\
        1\\
        0
      \end{smallmatrix}\right]}_{\begin{tabular}{c}
        \tiny world disturbance\\
        \tiny compensation
      \end{tabular}} + \underbrace{-\mathbf{d}_b\circ \left[\begin{smallmatrix}
        1\\
        1\\
        0
      \end{smallmatrix}\right]}_{\begin{tabular}{c}
        \tiny body disturbance\\
        \tiny compensation
      \end{tabular}},
    \end{split}
  \end{equation}
\end{block}

\column{0.48\textwidth} % Right column and width

\begin{block}{\small Linear MPC}
  \begin{equation}
    \scriptsize
    \begin{split}
      \mathbf{f}_d = &\overbrace{m_e\ddot{\mathbf{r}}_d}^{\begin{tabular}{c}
        \tiny reference \\
        \tiny feedforward
      \end{tabular}} + \overbrace{m_e\mathbf{c}_d}^{\begin{tabular}{c}
        \tiny MPC \\
        \tiny feedforward
      \end{tabular}} + \overbrace{m_eg\mathbf{\hat{e}}_3}^{\begin{tabular}{c}
        \tiny gravity \\
        \tiny compensation
      \end{tabular}} + \\
      &\underbrace{-\mathbf{d}_w\circ \left[\begin{smallmatrix}
        1\\
        1\\
        0
      \end{smallmatrix}\right]}_{\begin{tabular}{c}
        \tiny world disturbance\\
        \tiny compensation
      \end{tabular}} + \underbrace{-\mathbf{d}_b\circ \left[\begin{smallmatrix}
        1\\
        1\\
        0
      \end{smallmatrix}\right]}_{\begin{tabular}{c}
        \tiny body disturbance\\
        \tiny compensation
      \end{tabular}}.
    \end{split}
  \end{equation}

  \scriptsize \textbf{MPC feedforward:} desired acceleration from a constrained linear MPC.
\end{block}

\end{columns}

\fullciteinbox{baca2021mrs}{}

\end{frame}

%%}

\subsection{State estimators}

%%{ Estimator

\begin{frame}

\frametitle{The MRS UAV System}

  \begin{block}{MRS UAV System block diagram}
    \begin{figure}
      \begin{adjustbox}{max totalsize={1.0\textwidth}{.65\textheight}, center}
        \input{./fig/tikz/pipeline_diagram_estimator.tex}
      \end{adjustbox}
    \end{figure}
  \end{block}

  \fullciteinbox{baca2021mrs}{}

\end{frame}

\begin{frame}

\frametitle{The MRS UAV System}

  \begin{columns}[c]

  \column{0.55\textwidth} % Left column and width

  \begin{block}{Bank of filters}

    \begin{figure}
      \begin{adjustbox}{max totalsize={1.0\textwidth}{.65\textheight}, center}
        \input{./fig/tikz/bank_of_filters.tex}
      \end{adjustbox}
    \end{figure}

  \end{block}

  \column{0.40\textwidth} % Right column and width

  \begin{itemize}
    \item simultaneous estimation of UAV state in multiple frames of reference
    \item automatic \& manual switching of the main estimator
    \item automatic detection of estimator failures
    \item switching of an estimator is synchronizes throughout the control pipeline
  \end{itemize}

  \end{columns}

  \fullciteinbox{petrlik2020robust}{}

\end{frame}

%%}

\subsection{Reference generation}

%%{ MPC Tracker

%% | ----------------------- MPC Tracker ---------------------- |

\begin{frame}
\frametitle{The MRS UAV System}

  \begin{block}{MRS UAV System block diagram}
    \begin{figure}
      \begin{adjustbox}{max totalsize={1.0\textwidth}{.65\textheight}, center}
        \input{./fig/tikz/pipeline_diagram_tracker.tex}
      \end{adjustbox}
    \end{figure}
  \end{block}

  \fullciteinbox{baca2021mrs}{}

\end{frame}

\begin{frame}
\frametitle{Model Predictive Control Tracker}

\textbf{Feedback and Feed-forward multicopter controllers benefit from a smooth and feasible control reference. A step in desired position or velocity is not a feasible reference.}

\begin{columns}[c]

\column{0.48\textwidth} % Left column and width

\begin{block}{Problem}
  \begin{itemize}
    \item real-time reference generation
    \item \SI{100}{\hertz} full state reference for controllers: $\dot{\mathbf{x}}, \ddot{\mathbf{x}}, \dot{\ddot{\mathbf{x}}}$, $\ddot{\ddot{\mathbf{x}}}$, where $\mathbf{x} = \begin{pmatrix}
    r_x, r_y, r_z, \eta
    \end{pmatrix}^\intercal$ (pose and heading)
    \item UAV dynamics constraints satisfaction
    \item user input: trajectory consisting of poses sampled at regular intervals
  \end{itemize}
\end{block}

\column{0.48\textwidth} % Right column and width

\begin{block}{Solution}
  \begin{itemize}
    \item onboard real-time simulation of the linear translational UAV dynamics
    \item linear MPC (LCQP) solved at 100 Hz
    \item states of the simulated model sampled and taken as control reference
    \item \SI{8}{\second} prediction horizon used for inter-UAV collision avoidance
  \end{itemize}
\end{block}

\end{columns}

\fullciteinbox{baca2018model}{}

\end{frame}

\begin{frame}
\frametitle{Model Predictive Control Tracker}

\includegraphics[width=1.0\textwidth,trim={0 0.5cm 0 1.0cm},clip]{./fig/photos/tracker.jpg}

\begin{block}{Properties}

  \vspace{-1em}

  \begin{columns}[c]

  \column{0.48\textwidth} % Left column and width

  \begin{itemize}
    \item Linear MPC is solved in real time
    \item Mutual collision avoidance prevent damage during experimentation
  \end{itemize}

  \column{0.48\textwidth} % Right column and width
  \begin{itemize}
    \item The LCQP can be solved reliably
    \item The tracker can handle unfeasible trajectory references from a user
  \end{itemize}

  \end{columns}

\end{block}

% \fullciteinbox{baca2018model}{http://github.com/ctu-mrs/mrs_uav_trackers}
\fullciteinbox{baca2018model}{}

\end{frame}

%%}

%%{ control performance showcase

\begin{frame}
\frametitle{Showcase of the capabilities}

\only<1>{
  \begin{block}{Aggressive trajectory following}
    \begin{center}
      \mymovie[autostart,loop,start=30]{
        \includegraphics[width=0.7\textwidth]{./fig/mrs_uav_system_thumbnail.jpg}
      }{./videos/mrs_uav_system.mp4}\\
      Video: \url{http://mrs.felk.cvut.cz/mrs-uav-system}
    \end{center}
  \end{block}
}

\only<2>{
  \begin{block}{Flip}
    \begin{center}
      \mymovie[autostart,loop]{
        \includegraphics[width=0.8\textwidth]{./fig/flip_thumbnail.jpg}
      }{./videos/flip.mp4}\\
    \end{center}
  \end{block}
}

\end{frame}

%%}

\subsection{Trajectory generation}

%%{ Trajectory generation

\begin{frame}
\frametitle{Trajectory Generation}

  \begin{block}{MRS UAV System block diagram}
    \begin{figure}
      \begin{adjustbox}{max totalsize={1.0\textwidth}{.65\textheight}, center}
        \input{./fig/tikz/pipeline_diagram_trajectory_generation.tex}
      \end{adjustbox}
    \end{figure}
  \end{block}

  \fullciteinbox{baca2021mrs}{}

\end{frame}

\begin{frame}
\frametitle{Trajectory Generation}

\begin{block}{Path $\rightarrow$ Feasible Trajectory}
\includegraphics[width=0.9\textwidth]{./fig/trajectory_generation.jpg}
\end{block}

\begin{center}
  Fork of {\color{blue} ethz-asl/mav\_trajectory\_generation}
\end{center}

\fullciteinbox{richter2016polynomial}{}
\fullciteinbox{burri2015realtime}{}

\end{frame}

\begin{frame}
\frametitle{Trajectory Generation --- Improvements and beyond}

\begin{itemize}
  \item + smooth continuation of the prior UAV motion (Requires MPC Trackers)
  \item + Fallback solution if nlopt fails to optimize
  \item fixed poorly-implemented trajectory sampling
  \item + asynchronous time-outing
  \item recursive segment sub sectioning to meet desired corridor constraints
\end{itemize}

\fullciteinbox{baca2021mrs}{}

\end{frame}

\begin{frame}
\frametitle{Trajectory Generation}

\begin{block}{DARPA SubT Virtual - 2021}
  \begin{center}
    \mymovie[autostart,loop]{
      \includegraphics[width=0.8\textwidth]{./fig/darpa_virtual_cave_thumbnail.jpg}
    }{./videos/darpa_virtual_cave_short.mp4}\\
  \end{center}
\end{block}

\end{frame}

%%}

%% | --------------------- Implementation --------------------- |

\section{MRS UAV System --- Implementation}

%%{ Implementation diagram

\begin{frame}
\frametitle{MRS UAV System --- implementation}

\begin{columns}[c]

\column{0.48\textwidth} % Left column and width

\begin{block}{Modular system architecture}
  \begin{adjustbox}{max totalsize={1.0\textwidth}{.65\textheight}, center}
    \input{./fig/tikz/implementation_diagram.tex}
  \end{adjustbox}
\end{block}

\column{0.48\textwidth} % Right column and width

\begin{itemize}
  \item implementation in ROS1 (Noetic, Melodic)
  \item Flight software all in C++
  \item the system's core is nodeleted, running as a single process
  \item controllers as modules
  \item trackers as modules
\end{itemize}

\end{columns}

\begin{block}{Emphasis on safety and repeatability}
  \begin{itemize}
    \item self-checking system with fallbacks for most real-world faults
    \item protection of the UAV from the user
  \end{itemize}
\end{block}

\end{frame}

\begin{frame}
\frametitle{Full implementation diagram}

  \includegraphics[width=1.0\textwidth]{./fig/full_diagram.jpg}

\end{frame}

%%}

%%{ Software available

\begin{frame}
\frametitle{Software available as open-source}

\begin{block}{System provided as open source project as $>=$ 70 GitHub packages}
  \centering
  \Large\url{https://github.com/ctu-mrs/mrs_uav_system}
\end{block}

\begin{columns}[c]

\column{0.48\textwidth} % Left column and width

  \begin{block}{Available software}
    \begin{itemize}
      \item Core control pipeline
      \item Sensor drivers
      \item SLAMs
      \item Libraries and utilities
    \end{itemize}
  \end{block}

\column{0.48\textwidth} % Right column and width

  \begin{block}{Gazebo Simulation pipeline}
    \begin{itemize}
      \item Realistic simulation of our UAVs
      \item Dynamic UAV spawning \& de-spawning
      \item Modular sensor attachment mechanism
      \item Realistic modeling of common sensors
    \end{itemize}
  \end{block}

\end{columns}

\includegraphics[width=1.0\textwidth]{./fig/uavs.png}

\end{frame}

\begin{frame}
\frametitle{MRS UAV Platforms}

\begin{block}{Fly4Future --- customized platforms for research and development}
  \begin{center}
    \Large\url{https://dronebuilder.fly4future.com}
  \end{center}
\end{block}

\begin{block}{MRS group's UAVs are available}
  \includegraphics[width=1.0\textwidth]{./fig/uavs.png}
\end{block}

\end{frame}

%%}

%%{ Standalone Installation

\begin{frame}
\frametitle{System installation --- Research and development}

  \begin{block}{Visit the main entry point to install the system}
    \centering
    \Large\url{https://github.com/ctu-mrs/mrs_uav_system}
  \end{block}

  \begin{columns}[c]

  \column{0.48\textwidth} % Left column and width
  \begin{block}{{\color{red} Native installation}}
    \begin{itemize}
      \item not recommended for normal use
      \item allows changing and re-compiling the MRS UAV System
      \item might be more difficult to maintain up-to-date
      \item go-to for MRS members
    \end{itemize}
  \end{block}

  \column{0.48\textwidth} % Right column and width
  \begin{block}{Container installation}
    \begin{itemize}
      \item Requires Linux Kernel
      \item easier to use
      \item easier to keep up-to-date
      \item faster installation, lighter footprint on your system
      \item {\color{red} difficult to change the sources of the MRS UAV System}
    \end{itemize}
  \end{block}

  \end{columns}

\end{frame}

%%}

%%{ Additional apps

\begin{frame}
\frametitle{Additional software}

\begin{columns}[c]

\column{0.60\textwidth} % Left column and width
\begin{itemize}
  \small \item Bash --- We practically live there
  \small \item Vim --- Powerful terminal editor
  \small \item Tmux --- Terminal multiplexer
  \small \item SSH --- And how do you connect to the robot?
\end{itemize}

\column{0.38\textwidth} % Right column and width
\begin{itemize}
  \small \item Git --- Version control
  \small \item \texttt{C++} --- Anything real-time
  \small \item Python --- ANNs, computer vision
  \small \item Eigen --- Algebra in \texttt{C++}
\end{itemize}

\end{columns}

  \begin{block}{MRS Cheatsheet}
    \centering
    \Large\url{https://github.com/ctu-mrs/mrs_cheatsheet}
    \includegraphics[width=0.5\textwidth]{./fig/cheatsheet.jpg}
  \end{block}

\end{frame}

%%}

%%{ Summer School Installation

\begin{frame}
\frametitle{System installation --- Summer School 2022}

\begin{block}{Visit the Summer School 2022 repository}
  \centering
  \Large\url{https://github.com/ctu-mrs/summer-school-2022}
\end{block}

\begin{block}{What should you do?}
  \begin{itemize}
    \item (recommended) fork it
    \item clone it
    \item run \texttt{install.sh}
    \item (... work on the software in \texttt{mrim\_task/} ... ):
    \item run \texttt{simulation/run\_offline.sh} for lightweight emulation
    \item run \texttt{simulation/run\_simulation.sh} for Gazebo simulation
  \end{itemize}
\end{block}

\end{frame}

%%}

%%{ This presentation

\begin{frame}
\frametitle{Presentation pdf}

\begin{block}{Download the pdf from GitHub}
  \centering
  \Large\url{https://github.com/ctu-mrs/presentation_mrs_uav_system}
\end{block}

\end{frame}

%%}

\section{Summer School Task --- Installation}

%%{ Conclusion

\begin{frame}
  \frametitle{Conclusion}

  \begin{center}
    \huge Thanks for your attention\\
    \begin{center}
      \mymovie[autostart]{
        \includegraphics[width=0.7\textwidth]{./fig/flip_thumbnail.jpg}
      }{./videos/flip.mp4}\\
    \end{center}
  \end{center}

\end{frame}

%%}

%%{ References

\DeclareCiteCommand{\fullcite}
{\usebibmacro{prenote}}
{\clearfield{addendum}%
  \usedriver
  {\defcounter{minnames}{6}%
  \defcounter{maxnames}{6}}
{\thefield{entrytype}}}
{\multicitedelim}
{\usebibmacro{postnote}}

\begin{frame}[allowframebreaks]
  \frametitle{References}
  \tiny{
    \printbibliography[heading=none,title={}]
  }
\end{frame}

%%}

\end{document}
